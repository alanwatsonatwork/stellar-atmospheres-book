%!TEX root = main.tex

\chapter*{Preface}
\addcontentsline{toc}{chapter}{\numberline{}Preface}

\noindent
Dearly beloved, we are gathered here today to get through this thing called a graduate-level course on stellar atmospheres.

%These notes form the basis for a brief course in stellar atmospheres for graduate students.

In astronomy postgraduate program of the UNAM, I teach stellar atmospheres these as half of the stellar astrophysics course, with the other half being stellar structure and evolution. These are my notes, tidied up into book form.

Why do we still teach stellar astrophysics to astronomy graduate students? I give my students three answers:

\begin{enumerate}

\item
Most of what we know about stars we learn by studying their atmospheres.

I think stars per se are interesting, but the information we obtain from stars is also key to understanding galaxies.

\item
In undergraduate physics, radiation is a passive component; if it appears, is is typically assumed to be produced by matter emitting as a black body. 

In astrophysics, radiation is an active component, and the interplay between radiation and matter often determines the thermodynamic state of the matter. This is certainly the case for stellar atmospheres, but it is also a common occurrence in the interstellar medium, in circumstellar material, and in matter in the vicinity of a black hole. 

Thus, we can use stellar atmospheres as a laboratory for learning the physics of the interaction of matter and radiation, and then apply it in other areas of astrophysics.

\item 
If the first two answers fail to motivate my students, I remind them that they are going to need to pass the exam at the end of the course in order to proceed to the doctoral program.
\end{enumerate}
%!TEX root = main.tex

\noindent
Copyright {\copyright} 1999--2020 Alan M. Watson.

%\medskip
%\noindent
%Copying permitted for non-commercial educational purposes
%only. Alan M.\@ Watson asserts the moral right to be
%identifies as the author of these notes. In no way can he be
%held responsible for any liability with respect to these
%notes.

\bigskip
\bigskip

\noindent
Alan M.\ Watson\\
Instituto de Astronomía\\
Universidad Nacional Autónomo de México

\medskip
\noindent
Apartado Postal 70-264\\
04510 Coyoacán\\
Ciudad de México

\medskip
\noindent
alan@astro.unam.mx\\
http://www.astroscu.unam.mx/\raisebox{-3pt}{\~\relax}alan/

\bigskip
\bigskip

%\noindent
%\begin{tabular}{@{}l@{ }l@{ }l@{}}

% Team-taught with Will Henney in fall 1999. I taught the fundamentals,
% the grey atmosphere, and non-LTE atmospheres. Students: Teresa García
% and Carlos Rodríguez. Handwritten notes on yellow paper with blue ink.
%1st edition,&November 1999:&The blue-and-yellow edition.\\

% Team-taught with Paola D'Alessio in spring 2002. I taught the
% fundamentals, the grey atmosphere, and non-LTE atmospheres. Students:
% Carolina Durán, Arturo Godínez, and Mónica Rodríguez. The first LaTeX
% version, hence black-and-white.
%2nd edition,&May 2002:&The black-and-white edition.\\

% Team-taught with Paola D'Alessio in summer 2002. I taught the
% fundamentals, the grey atmosphere, and non-LTE atmospheres. Students:
% Luis Zapata. Chapter 1 was extensively rewritten. Only one student,
% hence the name.
%3rd edition,&June 2002:&The all-for-one edition.\\

% Team-taught with Paola D'Alessio in fall 2002. I taught the
% fundamentals, the grey atmosphere, LTE atmospheres, and non-LTE
% atmospheres. Students: Marna Albaran and, later, Luis Zapata.
%4th edition,&October 2002:&The continuity edition.\\

% Team-taught with Paola D'Alessio in fall 2003. I taught LTE and
% non-LTE atmospheres. Students: Ramiro Franco, Daniel Tafoya, and
% Martín Ávalos. Three very good students.
%5th edition,&December 2003:&The three-musketeers edition.\\

% Team-taught with Paola D'Alessio and Will Henney in spring 2005. Paula
% taught fundamentals, I taught LTE and non-LTE atmospheres, and Will
% taught winds (as Paola fell ill). Students: Edgar Ramírez, Alfonso
% Trejo, Idalia Hernández, Leticia Luis, Laura Gómez, Gabriela Montes,
% and Rosy Torres.
%5th edition,&May 2006:&The three-stooges edition.\\

% Team-taught with Will Henney in fall 2006. I taught fundamentals, the
% grey atmosphere, and LTE atmospheres. Students: Ramiro Alvarez,
% Roberto Galván, Arturo Gómez, Lizette Guzman, Francisco Hernández,
% Erendira Huerta, Selene Miranda, Manuel Neri, Citlali Neria, David
% Oseguera, Susana Pacheco, and Enrique Pérez. Marking the homeworks
% took one day each week.
%6th edition,&September 2006:&The eight-days-a-week edition.\\

% Team-taught with Will Henney in spring 2008. Students: Karla Alamo,
% Sergio Dzib, Alejandro González, José Vicente Hernández, Alfredo
% Manriquez, Eduardo Ortiz, Fátima Robles Valdez, Yetli Rosas Guevara, Jesus
% Toala, and Manuel Zamora Avilés. Carolina Durán marked the homeworks.
%7th edition,&February 2008:&The with-a-little-help-from-my-PhD-student edition.\\

% Team-taught with Leonid Georgiv in fall 2008.
%8th edition,&August 2008:&\\

%\end{tabular}

\setcounter{tocdepth}{1}
\tableofcontents

% $Id: chapter-3.tex 49 2008-09-09 23:13:36Z Alan $
%!TEX root = main.tex

\svnInfo $Id: chapter-3.tex 49 2008-09-09 23:13:36Z Alan $

\chapter{Simplifying Approximations}

\noindent
In this chapter we will introduce a number of widely-used approximations that can simplify considerably the solution of the equation of radiation transfer in atmospheres.

\section{Coherent and Isotropic Scattering}

Scattering is easy to incorporate into extinction -- we simply include
its contribution to the extinction coefficient -- but a general
treatment of the scattered emissivity is somewhat
involved; \citet[p.\ 5]{Schuster-1905}, in his pioneering
investigation of scattering in stellar atmospheres, notes that ``the
complete investigation leads to equations of such complexity that a
discussion becomes impossible''.

Given this, we  will make two
simplifying assumptions. First, we will assume that the scattering is
coherent, so that the scattering does not change the frequency of the
photon. Second, we will assume that the scattering is isotropic, so that
photons are emitted with equal probability into all directions. Under
these assumptions, the energy removed from the radiation field by
scattering in a volume $dV$, a frequency interval $(\nu,\nu+d\nu)$, and
a time interval $(t,t+dt)$ must be equal to the energy added to the
radiation field by scattering in the same volume $dV$, the same
frequency interval $(\nu,\nu+d\nu)$, and the same time interval
$(t,t+dt)$. From the definition of the extinction coefficient and the
emissivity, we then have
\begin{align}
\int_{4\pi}\!\!\!d\Omega\:
\sigma I_\nu
&=
\int_{4\pi}\!\!\!d\Omega\:
j_\nu^\mathrm{s}\\
4\pi\sigma J_\nu
&=
4\pi j_\nu^\mathrm{s}.
\end{align}
Thus, the scattered emissivity is given by
\begin{align}
j_\nu^\mathrm{s} = \sigma J_\nu.
\end{align}

\section{Local Thermodynamic Equilibrium}

In general, to determine the true emissivity $j_\nu^{e}$ and the
absorption coefficient $\alpha$, we need to determine the state of
matter in the atmosphere. As we will see, this is often quite difficult
because of the coupling between radiation and matter. For this reason,
we often make adopt the simplifying assumption of local thermodynamic
equilibrium (LTE).

In perfect thermodynamic equilibrium, we have detailed balance in all
processes. Detailed balance means that the rate of a process and its
inverse are equal, and follows from the symmetry of the laws of
microscopic physics under time reversals. If we consider absorption and
emission, in perfect thermodynamic equilibrium  we have
\begin{align}
j_\nu^\mathrm{e} = \alpha I_\nu.
\end{align}
However, we also have that $I_\nu = B_\nu$, and so we derive Kirchoff's
law, that in perfect thermodynamic equilibrium
\begin{align}
j_\nu^\mathrm{e} = \alpha B_\nu(T).
\end{align}
It's important to remember that is true only in perfect thermodynamic
equilibrium, and that stellar atmospheres are \emph{not} in perfect
thermodynamic equilibrium (as witnessed by, for example, the flow of
energy with them).

In the approximation of local thermodynamic equilibrium (LTE), we assume
that the matter has thermodynamic equilibrium properties at the local
temperature and density but the radiation does not. In particular, the
velocities of particles is assumed to be given by the Maxwell
distribution, the abundances of chemical species is assumed to be given
by the Saha distribution, and the populations of energy levels is
assumed to be given by the Boltzmann distribution, but we do \emph{not}
assume that the specific intensity is given by the Planck function. The
emissivity and extinction coefficient are determined by the state of
matter, so Kirchoff's law will still apply in LTE, and we still have
\begin{align}
j_\nu^\mathrm{e} = \alpha B_\nu(T).
\end{align}
It is vital to remember that these results are not true in general; in
order to obtain them, we had to assume LTE. Radiation emitted by matter
in LTE (or, more accurately, matter that is well-approximated by the LTE
approximation) is known as thermal radiation.

LTE significantly simplifies the solution of the equation
transfer, in that it reduces the state of matter at each
point to two variables -- the temperature and the density.
Nevertheless, the problem is still complicated, as the
temperature is determined by the interaction of radiation
with matter.

Note that LTE is an internally inconsistent approximation;
the occupation numbers will only have their true thermal
equilibrium values if the temperature is uniform and the
radiation field isotropic and Planckian. Both of these
requirements are violated in stellar atmospheres, in which
we often have large temperature gradients and in which the
radiation field is sharply peaked in the outward direction
and is non-Planckian. The LTE approximation in not too bad
for the velocity distribution of particles; collisions are
sufficiently frequent to maintain a single-temperature
Maxwellian distribution. However, it is much worse for the
ionization and excitation distributions. To be fully consistent, we would
have to adopt a completely non-LTE approach and directly
model the processes that create and destroy chemical species
and populate and depopulate energy levels. We will use this
approach in the later parts of this book, but for the
meantime we will adopt the LTE approximation.

In LTE we have $j_\nu^\mathrm{e} = \alpha B_\nu$, and so in general
the source function is 
\begin{align}
S_\nu \equiv \frac{j_\nu}{\chi} = \frac{\alpha B_\nu + j_\nu^\mathrm{s}}{\alpha + \sigma}.
\end{align}
Two important special cases of this are the absence of scattering, which
has $j_\nu^\mathrm{s} = 0$ and
\begin{align}
S_\nu = B_\nu,
\end{align}
and coherent, isotropic scattering, which has $j_\nu^\mathrm{s} =
\sigma J_\nu$ and 
\begin{align}
S_\nu = \frac{\alpha B_\nu + \sigma J_\nu}{\alpha +
\sigma}.
\label{equation-lte-and-coherent-and-isotropic-scattering}
\end{align}

\section{Radiative Equilibrium}

Stellar atmospheres are often assumed to be in
thermal equilibrium. By this we mean that the temperature at a given point is constant in time,
\begin{align}
\frac{\partial T(\vec r)}{\partial t} = 0.
\end{align}
In the plane-parallel approximation, this requires
that the total flux of energy be conserved throughout the atmosphere. A
special case of thermal equilibrium is \emph{radiative equilibrium}, in
which the flux of energy is carried entirely by radiation. In this case, in the
plane-parallel approximation, $F$ is constant throughout the atmosphere,
or in other words $dF/dz = 0$. The same radiative flux is found
entering the atmosphere from below, leaving it to above, and in all
intermediate layers.

The constancy of $F$ is a global statement of the condition of radiative
equilibrium in a plane-parallel atmosphere. We can also devise a local
statement: that the radiative energy absorbed in a small volume must be
equal to the radiative energy emitted by the same volume. If this were
not the case, the volume would suffer a net gain or loss of energy and
would rise or fall in temperature. From the definition of the extincion
coefficient, we can see that the total radiative heating rate per unit
volume is
\begin{align}
\int_0^\infty\!\!\!dv
\int_{4\pi}\!\!\!d\Omega
\chi I_\nu
=
4\pi
\int_0^\infty\!\!\!dv
\chi J_\nu,
\end{align}
where in substituting $J_\nu$ we have assumed that $\chi$ is
isotropic. Similarly, from the definition of the emissivity, the total
radiative cooling rate per unit volume is
\begin{align}
\int_0^\infty\!\!\!dv
\int_{4\pi}\!\!\!d\Omega\:
j_\nu = 4\pi \int_0^\infty\!\!\!dv\: j_\nu,
\end{align}
where again we have assumed that $j_\nu$ is isotropic.
In radiative equilibrium, we have the local condition
\begin{align}
\int_0^\infty\!\!\!dv\:
\chi J_\nu
=
\int_0^\infty\!\!\!dv\: j_\nu.
\end{align}
Note that in deriving the local condition we have not made any
assumption about the geometry of the atmosphere. Thus, the local
condition holds in any atmosphere in radiative equilibrium, whereas the
global condition that $F$ is constant is true only for a plane-parallel
atmosphere in radiative equilibrium. For example, in a spherically
symmetric atmosphere in radiative equilibrium, $F$ drops as $1/r^2$.

At first glance, it seems odd that in a plane-parallel atmosphere we
have two statements of the condition for radiative equilibrium:
$dF/d\tau = 0$ and $\int_0^\infty\!\!\!dv\:
\chi J_\nu
=\int_0^\infty\!\!\!dv\: j_\nu$. However, as we might expect, the
two are equivalent, and each implies the other. This is explored in
Problem~\ref{problem-radiative-equilibrium}.
